%!TEX encoding = UTF-8 Unicode

\section{Conclusions and Future Work}

Within the scope of cognitive robots that operate in unstructured environments, we have discussed a model that combines word affordance learning with body gesture recognition. We have proposed such an approach, based on the intuition that a robot can generalize its previously-acquired knowledge of the world~(objects, actions, effects, verbal descriptions) to the cases when it observes a human agent performing familiar actions in a shared \hr{} environment. We have shown promising preliminary results that indicate that a robot's ability to predict the future can benefit from incorporate the knowledge of a partner's action, facilitating scene interpretation and, as a result, teamwork.

In terms of future work, there are several avenues to explore. The main ones are (i)~the implementation of a fully probabilistic fusion between the affordance and the gesture components~(\eg, the soft decision discussed in Sec.~\ref{sec:combination}); (ii)~to run quantitative tests on larger corpora of \hr{} data; (iii)~to explicitly address the correspondence problem of actions between two agents operating on the same world objects~(\eg, a pulling action from the perspective of the human corresponds to a pushing action from the perspective of the robot, generating specular effects).
